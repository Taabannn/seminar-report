% !TeX root=../main.tex
% در این فایل، عنوان پایان‌نامه، مشخصات خود و چکیده پایان‌نامه را به انگلیسی، وارد کنید.

%%%%%%%%%%%%%%%%%%%%%%%%%%%%%%%%%%%%
\latinuniversity{University of Tehran}
\latincollege{College of Engineering}
\latinfaculty{School of Electrical Engineering}
\latindepartment{Communication Networks}
\latinsubject{Computer Engineering}
\latinfield{Algorithms and Computation}
\latintitle{A Deep Reinforcement Learning-Based Caching Strategy in IoT Networks}
\firstlatinsupervisor{Dr. Yazdani}
\secondlatinsupervisor{Dr. Shariatpanahi}
%\firstlatinadvisor{First Advisor}
%\secondlatinadvisor{Second Advisor}
\latinname{Taban}
\latinsurname{Soleymani}
\latinthesisdate{December 2022}
\latinkeywords{IoT Networks, Cache Replacement Policy, Deep Reinforcement Learning, Transient Data, Data Freshness}
\en-abstract{
The recent advances in Internet of Things (IoT) have brought about the importance of these networks more apparent. On the other hand, the explosive growth in the number of connected sensors and users, emphasizes on the importance of better networking schemes. Besides IoT networks have two unique attributes:
firstly, most IoT sensors are battery-powered and therefore they have limited energy to work with; secondly, the generated data by IoT sensors are generally valid for a limited duration
of time. In addition, conventional quality of service (QoS) measurements should be satisfied.
\paragraph{}
Caching transient data is a promising way that could fulfill the above requirements while bypassing the unique constraints of these networks.
Broadly speaking, a caching strategy can be applied on two different aspects:
it can affect the caching decision, i.e. whether or not an incoming content
chunk is to be stored in the cache, or it can affect the cache replacement decision, i.e. which piece of cached content to replace if a new content chunk is to be placed in a cache that has
reached its capacity.
Most proposed caching strategies only affect either the caching decision or the
cache replacement decision, with a few holistic approaches that take both into
consideration.
\paragraph{}
Our goal is to find a caching policy that reduces energy consumption in an IoT network while also considering the data freshness. On the other hand, the non-stationary behavior of files' popularity distribution is often a challenging attribute when we are working with traditional machine learning algorithms. Enabling deep reinforcement learning-based (DRL) agents helps us to develop efficient caching scheme without the need for any prior knowledge such as popularity distributions or the expiration time of files besides enhancing the cache hit rate and energy consumption of IoT networks.
}
