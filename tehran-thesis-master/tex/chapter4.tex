% !TeX root=../main.tex
\chapter{بحث و نتیجه‌گیری}
%\thispagestyle{empty} 
به طور کلی با توجه به ماهیت متمایز شبکه‌های اینترنت اشیا نسبت به سایر شبکه‌های مبتنی بر محتوا، استراتژی ذخیره‌سازی در این شبکه‌ها با چالش‌هایی روبه‌روست. از مهمترین چاش‌های پیش رو یکی مدت زمان محدود اعتبار داده و دیگری غیرایستان بودن توزیع محبوبیت فایل‌ها در محیط می‌باشد. علاوه بر این محیطی که استراتژی ذخیره‌سازی برای آن طراحی ‌می‌شود، در انتخاب الگوریتم و یا رویه‌ی ذخیره‌سازی اثر می‌گذارد. به عنوان مثال در صورتی‌که حرکت فزاره‌ها در شبکه‌های اینترنت اشیا قابل اغماض نباشد، فضای اکشن‌ها دیگر پیوسته نبوده و این مسئله تخمین ما از ارزش استیت‌ها و یا استیت-اکشن‌ها را دچار خطا می‌کند و بنابراین به طور مستقیم با استفاده از الگوریتم‌های متناظر به طور مستقیم سیاست بهینه را تخمین می‌زنیم. 

در محیط مسئله به تعداد گره‌های لبه در شبکه اینترنت اشیا، عامل تعریف می‌نماییم. هر چه میزان تنوع در محتویات ذخیره شده‌ی بین عامل‌ها تا حد معقولی بیشتر شود، عملاً استراتژی ذخیره‌سازی کارآمدتر خواهد بود. بدین منظور لازم است تا پردازنده‌ای داشته باشیم که به طور مشترک به گره‌های لبه برای ذخیره‌سازی یا جایگزینی محتوا در حافظه فرمان می‌دهد. حال هر چه این پردازنده به لایه‌ی کاربران نزدیکتر باشد یا حافظه‌ای برای ذخیره‌سازی محتوای مشترک داشته باشد،‌ عملاً استراتژی ذخیره‌سازی موفقیت‌آمیزتر خواهد بود که برای تحقق این امر می توان به روش‌های مختلف گره‌های در لایه‌ی لبه را خوشه‌بندی کرد.