% !TeX root=../main.tex
% در این فایل، عنوان پایان‌نامه، مشخصات خود، متن تقدیمی‌، ستایش، سپاس‌گزاری و چکیده پایان‌نامه را به فارسی، وارد کنید.
% توجه داشته باشید که جدول حاوی مشخصات پروژه/پایان‌نامه/رساله و همچنین، مشخصات داخل آن، به طور خودکار، درج می‌شود.
%%%%%%%%%%%%%%%%%%%%%%%%%%%%%%%%%%%%
% دانشگاه خود را وارد کنید
\university{دانشگاه تهران}
% پردیس دانشگاهی خود را اگر نیاز است وارد کنید (مثال: فنی، علوم پایه، علوم انسانی و ...)
\college{پردیس دانشکده‌های فنی}
% دانشکده، آموزشکده و یا پژوهشکده  خود را وارد کنید
\faculty{دانشکده‌ی مهندسی برق و کامپیوتر}
% گروه آموزشی خود را وارد کنید (در صورت نیاز)
\department{گروه شبکه و سیستم‌های امن}
% رشته تحصیلی خود را وارد کنید
%\subject{مهندسی برق}
% گرایش خود را وارد کنید
%\field{شبکه‌های مخابراتی}
% عنوان پایان‌نامه را وارد کنید
\title{استراتژی ذخیرە‌سازی در شبکه‌های اینترنت اشیا براساس یادگیری تقویتی عمیق}
% نام استاد(ان) راهنما را وارد کنید
\firstsupervisor{دکتر ناصر یزدانی}
\firstsupervisorrank{استاد}
\secondsupervisor{دکتر سید پویا شریعت پناهی}
\secondsupervisorrank{استاد}
% نام استاد(دان) مشاور را وارد کنید. چنانچه استاد مشاور ندارید، دستورات پایین را غیرفعال کنید.
%\firstadvisor{دکتر مشاور اول}
%\firstadvisorrank{استادیار}
%\secondadvisor{دکتر مشاور دوم}
% نام داوران داخلی و خارجی خود را وارد نمایید.
%\internaljudge{دکتر داور داخلی}
%\internaljudgerank{دانشیار}
%\externaljudge{دکتر داور خارجی}
%\externaljudgerank{دانشیار}
%\externaljudgeuniversity{دانشگاه داور خارجی}
% نام نماینده کمیته تحصیلات تکمیلی در دانشکده \ گروه
%\graduatedeputy{دکتر نماینده}
%\graduatedeputyrank{دانشیار}
% نام دانشجو را وارد کنید
\name{تابان}
% نام خانوادگی دانشجو را وارد کنید
\surname{سلیمانی}
% شماره دانشجویی دانشجو را وارد کنید
\studentID{810100377}
% تاریخ پایان‌نامه را وارد کنید
\thesisdate{دی 1401}
% به صورت پیش‌فرض برای پایان‌نامه‌های کارشناسی تا دکترا به ترتیب از عبارات «پروژه»، «پایان‌نامه» و «رساله» استفاده می‌شود؛ اگر  نمی‌پسندید هر عنوانی را که مایلید در دستور زیر قرار داده و آنرا از حالت توضیح خارج کنید.
\projectLabel{گزارش نهایی درس سمینار}

% به صورت پیش‌فرض برای عناوین مقاطع تحصیلی کارشناسی تا دکترا به ترتیب از عبارت «کارشناسی»، «کارشناسی ارشد» و «دکتری» استفاده می‌شود؛ اگر نمی‌پسندید هر عنوانی را که مایلید در دستور زیر قرار داده و آنرا از حالت توضیح خارج کنید.
%\degree{ن}
%%%%%%%%%%%%%%%%%%%%%%%%%%%%%%%%%%%%%%%%%%%%%%%%%%%%
%% پایان‌نامه خود را تقدیم کنید! %%
\dedication
{
{\Large تقدیم به:}\\
\begin{flushleft}{
	\huge
	همسر و فرزندانم\\
	\vspace{7mm}
	و\\
	\vspace{7mm}
	پدر و مادرم
}
\end{flushleft}
}
%% متن قدردانی %%
%% ترجیحا با توجه به ذوق و سلیقه خود متن قدردانی را تغییر دهید.
\acknowledgement{
سپاس خداوندگار حکیم را که با لطف بی‌کران خود، آدمی را به زیور عقل آراست.

در آغاز وظیفه‌  خود  می‌دانم از زحمات بی‌دریغ اساتید  راهنمای خود،  جناب آقای دکتر ... و ...، صمیمانه تشکر و  قدردانی کنم که در طول انجام این پایان‌نامه با نهایت صبوری همواره راهنما و مشوق من بودند و قطعاً بدون راهنمایی‌های ارزنده‌ ایشان، این مجموعه به انجام نمی‌رسید.

از جناب آقای دکتر ... که  زحمت مشاوره‌، بازبینی و تصحیح این پایان‌نامه را تقبل فرمودند کمال امتنان را دارم.

%از همکاری و مساعدت‌های دکتر ... مسئول تحصیلات تکمیلی و سایر کارکنان دانشکده بویژه سرکار خانم ... کمال تشکر را دارم.

با سپاس بی‌دریغ خدمت دوستان گران‌مایه‌ام، خانم‌ها ... و آقایان ... در آزمایشگاه ...، که با همفکری مرا صمیمانه و مشفقانه یاری داده‌اند.

و در پایان، بوسه می‌زنم بر دستان خداوندگاران مهر و مهربانی، پدر و مادر عزیزم و بعد از خدا، ستایش می‌کنم وجود مقدس‌شان را و تشکر می‌کنم از خانواده عزیزم به پاس عاطفه سرشار و گرمای امیدبخش وجودشان، که بهترین پشتیبان من بودند.
}
%%%%%%%%%%%%%%%%%%%%%%%%%%%%%%%%%%%%
%چکیده پایان‌نامه را وارد کنید
\fa-abstract{
پیشرفت‌های اخیر در حوزه‌ی شبکه‌های اینترنت اشیا، اهمیت این حوزه را بیش از پیش نمایان ساخته است. از سوی دیگر با رشد چشمگیر تعداد حسگرها و کاربران متصل، نیاز شدید به یک رویه‌ی هوشمند در شبکه دیده می‌شود. گذشته از این، شبکه‌های اینترنت اشیا دارای دو مشخصه‌ی متمایز می‌باشند: اولاً تأمین انرژی اکثر حسگرها به وسیله‌ی باتری انجام می‌شود یا به عبارتی سطح انرژی حسگرها محدود است و ثانیاً داده‌ی تولید شده توسط حسگرها در شبکه‌ی اینترنت اشیا تا مدت محدودی معتبر است. به علاوه باید در نظر داشت که معیارهای کیفیت سرویس نیز باید در این شبکه‌ها اقناع شوند.

ذخیره‌سازی داده‌ی گذرا با در نظر گرفتن قیود موجود در شبکه‌های اینترنت اشیا، به عنوان یک راه اثربخش برای اقناع معیارهای کیفیت سرویس در شبکه شناخته می‌شود. عموماً استراتژی ذخیره سازی بر یکی از دو جنبه‌ی متفاوت مسئله‌ اعمال می‌شوند: این استراتژی می‌تواند بر معیار تصمیم‌گیری برای ذخیره‌ی یک محتوا اثر بگذارد و یا اینکه با توجه به محدودیت حافظه‌ی ذخیره‌سازی، در فاز جایگزینی محتوا اثرگذار باشد. البته تعدادی از روش‌های کل نگرانه هر دو جنبه را مورد مطالعه قرار می‌دهند. 

هدف اصلی ما یافتن سیاست بهینه‌ی ذخیره‌سازی با در نظر داشتن قید تازگی داده، برای کاهش مصرف انرژی در شبکه‌های اینترنت اشیا می‌باشد. از سوی دیگر متغیر بودن توزیع محبوبیت محتوا با زمان،  به کارگیری الگوریتم‌های سنتی یادگیری ماشین را با چالش رو به رو کرده است. فعالسازی عامل‌های هوشمند مبتنی بر یادگیری تقویتی عمیق، به ما در گسترش رویه‌ی ذخیره سازی کارآمد، بدون نیاز به دانستن توزیع محبوبیت فایل‌ها و زمان انقصا کمک می‌کند و در عین حال نرخ موفقیت در ذخیره سازی و مصرف انرژی را بهبود می‌دهد.
}
% کلمات کلیدی پایان‌نامه را وارد کنید
\keywords{شبکه‌های اینترنت اشیا، سیاست جایگزینی محتوا در حافظه، یادگیری تقویتی عمیق، داده‌ی گذرا، تازگی داده}
% انتهای وارد کردن فیلد‌ها
%%%%%%%%%%%%%%%%%%%%%%%%%%%%%%%%%%%%%%%%%%%%%%%%%%%%%%
